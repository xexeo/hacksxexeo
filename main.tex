\documentclass{article}
\usepackage[utf8]{inputenc}
\usepackage[show]{comentario} % \usepackage[hide]{comentario} para ocultar os comentários

% vamos usar o biblatex, recomendação
%citestyle=alphabetic,bibstyle=authortitle
%\usepackage[citestyle=authoryear,articlein=false,
%style=ext-authoryear-comp,natbib=true]{biblatex}
%\bibliographystyle{Estilos/coppe-plain}
%\addbibresource{biblio.bib}

\title{testes de comentario}
\author{Eduardo Mangeli}
\date{October 2019}

\begin{document}

\listofcomentario

\maketitle

\section{Introduction}

%----- instruções de uso
%
%  copiar o arquivo comentario.sty para o diretório raiz do projeto
%
%  importar o pacote no arquivo principal usando o comando 
%  \usepackage{comentario} no preâmbulo do documento
%
%  comentar o texto com o comando abaixo:
%  \comentario[cor opcional]{comentário}{texto do documento}
%
%  para ocultar os comentários usar a opção hide no comando inicial
%  \usepackage[hide]{comentario} no preâmbulo do documento



Nam dui ligula, fringilla a, euismod sodales, sollicitudin vel, wisi.  Morbi
auctor lorem non justo. Nam lacus libero, pretium at, lobortis vitae, ultricies et,
tellus. Donec aliquet, tortor sed accumsan bibendum, erat ligula aliquet magna,
vitae \comentario{isso vai ficar comentado. A citação precisa ficar entre \{ \}}{ornare odio metus {\cite{Juran2010}} a mi. Morbi ac orci et nisl hendrerit mollis.} \comentario[red]{Esse é um comentário muito longo para testar o truncate, e você não pode pedir para ele ser menor}{Suspendisse}
ut massa. Cras nec ante. Pellentesque a nulla. Cum sociis natoque penatibus et
magnis dis parturient \comentario[magenta]{um comentário de um texto com nota de rodapé}{montes\footnote{Só uma nota de rodapé real de exemplo}}, nascetur ridiculus mus. Aliquam tincidunt urna.
Nulla ullamcorper \comentario[green]{esse aqui é um outro comentario mudando a cor}{vestibulum} turpis. Pellentesque cursus \comentario[cyan]{A cor padrão é amarelo mas outras podem ser usadas}{luctus mauris}.




Parece ser uma questão bem interessante. De toda forma, esperar mais que isso não me parece ser muito interessante. Essa coisas nem sempre acontecem quando queremos e isso não parece ser bom. Contudo, quanto mais próximos do que queremos certamente melhores estaremos.

\bibliographystyle{apalike}
\bibliography{biblio.bib}
%\printbibliography

\end{document}
